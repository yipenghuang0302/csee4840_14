\subsection{Timing Design}

\subsubsection{Submodule Timing Design}

The FPGA has a limited number of DSPs which are used to implement multipliers, so we must time multiplex their use. We schedule the use of our functional units so that parts of the inverse kinematics algorithm can run in parallel, and so that functional units can be reused in different parts of the algorithm.

Figure \ref{fig:schedule} shows the timing design of the accelerator. The diagram lists the functional unit hardware resources along the vertical axis. These include the sine and cosine unit, array multiplier, matrix multiplier, array divider, and square root units. The clock cycles each of these resources are active are shown along the horizontal axis. The top half of the diagram shows the first ~110 clock cycles of the algorithm, which is dedicated to finding the forward kinematics Jacobian matrix. The bottom half of the diagram is the second ~140 clock cycles of the algorithm, which carries out the damped least squares algorithm.

% Figure
\begin{figure}[ht]
\center
\epsfig{figure=../figures/schedule.eps,
width=1.0\columnwidth}
\caption{The timing design of the accelerator. The diagram lists the functional unit hardware resources along the vertical axis, and displays which clock cycles those resources are active along the horizontal axis. In each cell we provide a brief note on which part of the algorithm occurs in that cycle.}
\label{fig:schedule}
\end{figure}

While it may appear that there is minimal parallelism in the damped least squares algorithm, we point out that the matrix multiplier, divider, and adder units are all parallel, SIMD-style functional units. The matrix multiplier in particular does 36 parallel multiplications at once.

\subsubsection{Control Signal Timing Design}

Figure \ref{fig:timing} shows the control and data signal timing for the accelerator. The host computer writes the target and initial joint angles to the accelerator, followed by the enable signal that launches calculation. Once updated joint angles are available, the accelerator asserts the done signal. The host computer can then read out the new angles and prepare for the next iteration.

% Figure
\begin{figure}[ht]
\center
\epsfig{figure=../figures/timing.eps,
width=0.8\columnwidth}
\caption{The control and data signal timing for the accelerator.}
\label{fig:timing}
\end{figure}