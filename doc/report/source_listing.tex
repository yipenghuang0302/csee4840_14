\section{Source Listings}
Complete listings of every file you wrote for the project. Include C source, SystemVerilog source, and things such as .mhs files. Don't include any file that was generated automatically.

\subsection{Hardware}
\begin{verbatim}
SoCKit_top : top-level file
ik_interface
ik_swift_36
full_jacobian
full_mat
t_block
inverse
cholesky
lt_inverse
mat_mult
array_mult
\end{verbatim}

\subsection{Software}


\lstset{ %
	backgroundcolor=\color{white},   % choose the background color; you must add \usepackage{color} or \usepackage{xcolor}
	%basicstyle=\footnotesize,       % the size of the fonts that are used for the code
		basicstyle=\scriptsize\ttfamily,         % the size of the fonts that are used for the code
		breakatwhitespace=false,         % sets if automatic breaks should only happen at whitespace
		breaklines=true,                 % sets automatic line breaking
		captionpos=b,                    % sets the caption-position to bottom
		commentstyle=\color{mygreen},    % comment style
		deletekeywords={...},            % if you want to delete keywords from the given language
		escapeinside={\%*}{*)},          % if you want to add LaTeX within your code
		extendedchars=true,              % lets you use non-ASCII characters; for 8-bits encodings only, does not work with UTF-8
		frame=single,                    % adds a frame around the code
		keepspaces=true,                 % keeps spaces in text, useful for keeping indentation of code (possibly needs columns=flexible)
		keywordstyle=\color{blue},       % keyword style
		language={[Objective]Caml},      % the language of the code
		morekeywords={*,...},            % if you want to add more keywords to the set
		numbers=left,                    % where to put the line-numbers; possible values are (none, left, right)
		numbersep=5pt,                   % how far the line-numbers are from the code
		numberstyle=\tiny\color{mygray}, % the style that is used for the line-numbers
		rulecolor=\color{red},           % if not set, the frame-color may be changed on line-breaks within not-black text (e.g. comments (green here))
		showspaces=false,                % show spaces everywhere adding particular underscores; it overrides 'showstringspaces'
		showstringspaces=false,          % underline spaces within strings only
		showtabs=false,                  % show tabs within strings adding particular underscores
		stepnumber=5,                    % the step between two line-numbers. If it's 1, each line will be numbered
		stringstyle=\color{mymauve},     % string literal style
		tabsize=2,                       % sets default tabsize to 2 spaces
		title=\lstname                   % show the filename of files included with \lstinputlisting; also try caption instead of title
}





\lstinputlisting{../../software/CConfigLoader.h}
\lstinputlisting{../../software/CConfigLoader.cpp}
\lstinputlisting{../../software/Makefile}
\lstinputlisting{../../software/lexer.cpp}
\lstinputlisting{../../software/lexer.h}
\lstinputlisting{../../software/hps.h}
\lstinputlisting{\detokenize{../../software/hps_0.h}}
\lstinputlisting{../../software/SharedTypes.h}
\lstinputlisting{\detokenize{../../software/robot_arm.cpp}}
\lstinputlisting{\detokenize{../../software/xml_help/pugiconfig.hpp}}
\lstinputlisting{\detokenize{../../software/xml_help/pugixml.cpp}}
\lstinputlisting{\detokenize{../../software/xml_help/pugixml.hpp}}
\lstinputlisting{\detokenize{../../software/ik_driver.c}}
\lstinputlisting{\detokenize{../../software/ik_driver.h}}
\lstinputlisting{../../software/robots/robot.xml}
