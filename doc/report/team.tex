\subsection{Team Member Roles}

\paragraph{Yipeng} served the role of system integrator for the hardware. He created the sine and cosine hardware, along with the hardware for finding the forward kinematics matrix. He validated the hardware as a whole and its subcomponents individually against software models. Yipeng tuned the hardware timing and precision, in order to guarantee the solution converges while meeting FPGA area requirements.

\paragraph{Lianne} created the skeleton of the C implementation of our algorithm used 
for understanding the algorithm and how components would fit together. She 
produced the template OpenGL file which was used to demonstrate that the 
IKSwift accelerator was working. Lianne also looked into matrix inverse 
algorithms and determined the best method for inverting matrices in our 
program. She implemented the two step process of inverting matrices in 
SystemVerilog so that it could be interfaced with the rest of the 
hardware files. After our hardware was built she worked with Qsys to 
develop the files required to interface the IKSwift accelerator with 
the device driver running on the ARM processor. 

\paragraph{Richard} designed the skeleton for the SystemVerilog module that calculates the Jacobian
for a given configuration. Other than that, all of Richard's tangible contributions were on the
software end of the project. He designed the device driver to communicate with the hardware and
modified Lianne's openGL template to display an actual robot arm and interact with the device
driver. In terms of deliverables, Richard was in charge of maintaining accurate milestones over the
course of the semester and developed the pseudocode to describe the Jacobian and
damped least-squares algorithms.
