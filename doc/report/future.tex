\subsection{Lessons Learned}

\paragraph{Yipeng}

Good software engineering practices are also good for hardware. We wrote models and tests first to convince ourself the algorithm would work, and then exhaustively show each module gives acceptably accurate results. The last bugs will be hiding in code the team hasn't tested.

Minimize the known unknowns so there's time left to deal with unknown unknowns. We aimed to make the design as clear as possible to the team, if not on paper too. This allowed us to estimate timing and area costs even before coding, which helped us minimize revisions.

\paragraph{Lianne}

The most important lesson I learned for programming embedded systems or any hardware design 
is that planning first is absolutely necessary for not just a good design but one that works. 

By writing out detailed timing diagrams and just matching up what step of an algorithm has to 
happen when is crucial to identifying what resources can be shared and what can happen in 
parallel. 

\paragraph{Richard}

My strongest suggestion would be to have weekly meetings that all members of your group attend.
At these meetings we talked about any progress we made or obstacles we ran into over
the past week. We would then discuss what the next logical step of our work would entail, and assign
portions of this work to the different group members. Furthermore, we would make sure that our
current project status correctly coincided with the milestones that should have been completed
by that point. Luckily, we were always a bit ahead of the milestones we gave for ourselves,
so that by the end of the semester we had some leeway and didn't feel stressed. This is not
an endorsement for trivial milestones, but instead a suggestion that groups always try to stay
a step ahead of their own expectations; it will make the project much easier in the end.