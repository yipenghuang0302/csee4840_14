\section{Hardware/Software Interface}

\subsection{Software Driver}

We designed a ``joint" peripheral through which the software and 
hardware components of our system interact. 
The driver for the joint peripheral provides an ioctl that copies a struct to and from the 
user with the following components:

\begin{center}
	\begin{tabular}{|c|c|}
		\hline
		Field & Comments \\
		\hline
		\texttt{unsigned int start\_signal} & signal for HW to run an iteration of the algorithm\\
		\hline
		\texttt{unsigned int done\_signal} & signal for SW to run an iteration of the openGL display loop\\
		\hline
		\texttt{signed int target[3]} &  The x,y,z, coordinates of the target position 
																						for our end-effector \\
		\hline
		\texttt{signed char joint} &  Indicate which joint we're getting/setting\\
		\hline
		\texttt{unsigned char joint\_type} & The $i$th bit is 1 if the $i$th joint is rotational, 
																					0 for translational \\
		\hline
		\texttt{unsigned int magnitude} & ~\\
		\hline
	\end{tabular}
\end{center}
After displaying the most recent configuration of the robot appendage, the software will set
\texttt{start\_signal} to 1 to notify the hardware to run another iteration of the algorithm. The
software will then spin until \texttt{done\_signal} is set to 1, indicating that the D-H parameters have been
updated and the appendage needs to be redisplayed.
The \texttt{joint} field represents which joint we're referring to 
(we have a total of \texttt{MAX\_JOINT} joints), the \texttt{joint\_type}
field keeps track of the type of every joint in system,  
and the \texttt{magnitude} field holds the value of the theta parameter for the joint in question (all other parameters are hard coded
		into the hardware).
We also use the \texttt{joint} field to indicate when we're setting non-joint values. If
$\texttt{joint} = -1$, then we're setting the \texttt{target} value, and if $\texttt{joint} = -2$,
then we're setting the \texttt{start\_signal} value.

\subsection{Hardware Register Set}

The registers used by the device driver are represented with the following struct, following the example given in lab 3:

\begin{verbatim}
struct joint_dev{
	resource_size_t start; /* Address of start of registers */
	resource_size_t size; /* Size of registers */
	void __iomem *virtbase; /* Pointer to registers */
} dev;
\end{verbatim}

We chose to have a fairly bare struct, as all of our interactions with the registers occur with ioreads and iowrites that interact
directly with memory.

\subsection{Top-Level Hardware Interface}

\begin{verbatim}
module ikswift_top (
	input logic clk, rst, en;
	input logic [5:0] joint_type;
	input logic target_0 [15:0];
	input logic target_1 [15:0];
	input logic target_2 [15:0];
	input logic target_3 [15:0];
	input logic target_4 [15:0];
	input logic target_5 [15:0];
	input logic dh_param_0 [3:0] [15:0];
	input logic dh_param_1 [3:0] [15:0];
	input logic dh_param_2 [3:0] [15:0];
	input logic dh_param_3 [3:0] [15:0];
	input logic dh_param_4 [3:0] [15:0];
	input logic dh_param_5 [3:0] [15:0];
	output logic done;
	output logic output_0 [15:0];
	output logic output_1 [15:0];
	output logic output_2 [15:0];
	output logic output_3 [15:0];
	output logic output_4 [15:0];
	output logic output_5 [15:0];
);
\end{verbatim}
