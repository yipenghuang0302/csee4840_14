The general inverse kinematics algorithm relies on the following concepts, which we present in order:

Homogeneous Transformations: Kinematics operates on frames, which are as set of axes and coordinates that describe space. Frames can be global or local. A local frame would be useful in describing to x, y, z positions of an object in space, along with the orientation (direction it is pointing) in space. One can transform from one frame to the next using homogeneous transforms, which are described as 4 by 4 matrices. For background information on homogeneous transforms refer to \cite{frames}

Kinematic Chain:
Each joint of a robot has a coordinate frame; by convention without exception, revolute joints are represented as a rotation about axis Z, and prismatic joints are translations along axis Z. A collection of joints on an actuator form a kinematic chain.

Denavit-Hartenberg Parameters:
The standard notation for describing an actuator is to first describe the rotation about the Z axis by a joint angle, translate along by the Z axis by the link offset, translate along the X axis by the link length, and rotate by the X axis by link twist.

fully acuated
Figure 12–1. ALTFP_ATAN Ports


