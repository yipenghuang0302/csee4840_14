\subsection*{Motivation}

Computers are increasingly embedded in the real world, often permanently attached to sensors and actuators. An example of such a system are robots. Computer systems that must coordinate with sensors and actuators are distinct from general purpose computers, and the desired hardware to support its applications will also be different.

Specialty hardware such as GPUs can support hard-hitting sensing algorithms, especially those that involve image processing. Equally important, but less well studied, is how computer hardware should adapt to support controlling actuators. Actuator algorithms have yet to appear in well known computer architecture research workloads. Early studies of computer architecture support for robotics show that general purpose CPUs suffer when running typical robotic workloads\cite{Caselli}.

Problems that arise when controlling actuators, such as kinematics, dynamics, obstacle avoidance, collision detection, have been found to occupy a large portion of computer runtime in robotics. Live measurements show 33-66 percent dedicated to embedded computer on robot. In particular, the inverse kinematic problem is interesting because it has features of two distinct workload categories: sparse matrix math and graph traversal. This project will focus on building hardware for inverse kinematics.

