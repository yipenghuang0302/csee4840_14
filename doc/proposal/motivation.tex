Motivation:
A robot can be thought of as a system of sensors, computers, and actuators. The onboard computer supports the software needs for sensing and actuation, but the hardware resources of of mobile robots have consistently been a bottleneck for building robots that operate in real time. Specialty hardware such as GPUs can support hard-hitting sensing algorithms, especially those that involve image processing. In terms of other hot robotics algorithms, planning has yet to receive significant hardware support. In general, planning algorithms belong to either mobility or manipulator planning. Mobility concerns the decisions that have to made in rolling on a mobile base, and involve memory-intensive tree search algorithms such such as
 		collision detection	
			nearest neighbor search
			point cloud library
			iterative closest point (found to be heavy)
			k-d tree
		depth sorting
		occupancy
Manipulators require IK, which are a distinct workload because the algorithms combine the characteristics of graph traversal and sparse matrix.

full system robot profiling
anecdotal evidence from robotics researchers
Berkeley dwarfs identified to be distinct workloads, yet to be covered in commercial benchmarks
computer architecture research methods such as instrumentation and profiling

live measurements show 33-66 percent dedicated to embedded computer on robot,
15-29 % to the microcontrollers
