\section{Hardware/Software Interface}

\subsection{Software Driver}

We will design a ``joint" peripheral through which the software and hardware components of our system will interact. 
The driver for the joint peripheral will provide an ioctl that copies a struct to and from the user with the following components:

\begin{center}
	\begin{tabular}{|c|c|}
		\hline
		Field & Comments \\
		\hline
		\texttt{unsigned char target[3]} &  The x,y,z, coordinates of the target position for our end-effector \\
			\hline
		\texttt{unsigned char joint} &  0,1,...,JOINT\_DOF-1 \\
			\hline
		\texttt{unsigned char joint\_type} & The $i$th bit is 1 if the $i$th joint is rotational, 0 for translational \\
			\hline
		\texttt{unsigned char parameter} & THETA,L\_OFFSET,L\_DISTANCE,ALPHA \\
			\hline
		\texttt{unsigned short magnitude} & ~\\
		\hline
	\end{tabular}
\end{center}
The \texttt{joint} field represents which joint we're referring to (we have a total of \texttt{JOINT\_DOF} joints), the \texttt{joint\_type}
field keeps track of the type of every joint in system, the \texttt{parameter} field
is a constant signifying which D-H parameter we're reading or writing, and the magnitude field holds the value of that D-H parameter.

\subsection{Hardware Register Set}

The registers used by the device driver are represented with the following struct, following the example given in lab 3:

\begin{verbatim}
struct joint_dev{
	struct resource res;		/* Resource: our registers */
	void __iomem *virtbase; /* Pointer to registers */
	u8 joint_type; /* ith bit is 1 if ith joint is rotational, 0 for translational */
	u16 target[3]; /* Target position */
	u16 dh_params[JOINT_DOF * 4] /* Every joint has 4 parameters */
} dev;
\end{verbatim}

The \texttt{joint\_type} and \texttt{target} fields mirror the same fields in the user-level struct and are only set once during the configuration
stage of our pipeline. The \texttt{dh\_params} array in the given struct will be used to read and write magnitudes for various D-H parameters in the 
given configuration. For example, the magnitude of the third joint's theta parameter would be stored at \texttt{dh\_params[2*4 + THETA]}. Note that
since the only parameters that our algorithm modifies are $\theta_i$ and $d_i$ for joint $i$, these are the only components of our struct that
will be modified after the start of our algorithm.

\subsection{Top-Level Hardware Interface}

\begin{verbatim}
module ikswift_top (
	input logic clk, rst, en;
	input logic [5:0] joint_type;
	input logic target_0 [15:0];
	input logic target_1 [15:0];
	input logic target_2 [15:0];
	input logic target_3 [15:0];
	input logic target_4 [15:0];
	input logic target_5 [15:0];
	input logic dh_param_0 [3:0] [15:0];
	input logic dh_param_1 [3:0] [15:0];
	input logic dh_param_2 [3:0] [15:0];
	input logic dh_param_3 [3:0] [15:0];
	input logic dh_param_4 [3:0] [15:0];
	input logic dh_param_5 [3:0] [15:0];
	output logic done;
	output logic output_0 [15:0];
	output logic output_1 [15:0];
	output logic output_2 [15:0];
	output logic output_3 [15:0];
	output logic output_4 [15:0];
	output logic output_5 [15:0];
);
\end{verbatim}