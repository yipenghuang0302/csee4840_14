\subsection{Custom Functional Units}
In this section we describe the building blocks of the accelerator—our custom functional units—which we assemble to create the submodules described in previous sections. These custom functional units are assembled from IP designs generated by Altera MegaFunctions.

\subsubsection{16-bit Sine Cosine}
We use Taylor series to estimate sine and cosine functions. The Taylor series expansion for sine is $sin(x) = x - x^3/3! + x^5/5!$... Figure \ref{fig:sine} shows the effect of including more terms on accuracy. 

Similarly, the Taylor series expansion for cosine is $cos(x) = 1 - x^2/2! + x^4/4!$... Figures  \ref{fig:sine}, \ref{fig:cosine} show the effect of including more terms on accuracy.

% Figure
\begin{figure}[ht]
\center
\epsfig{figure=../figures/sine.eps,
width=0.6\columnwidth}
\caption{The Taylor series estimate of sine becomes more accurate as more terms are added.}
\label{fig:sine}
\end{figure}

% Figure
\begin{figure}[ht]
\center
\epsfig{figure=../figures/cosine.eps,
width=0.6\columnwidth}
\caption{The Taylor series estimate of cosine becomes more accurate as more terms are added.}
\label{fig:cosine}
\end{figure}

For our design, we will use just two terms to estimate sine and three terms to estimate cosine. This level of accuracy is acceptable because the robot only moves small increments at each iteration of the inverse kinematics algorithm, and low order estimates for sine and cosine suffice for function values near the origin.



\subsubsection{4x4 Matrix Matrix Multiplication}
4x4 matrix multiply pipeline diagram
\subsubsection{6x6 Matrix Matrix Multiplication}
6x6 matrix multiply pipeline diagram
\subsubsection{6x6 Matrix 6x1 Vector Multiplication}
6x6 6x1 matrix vector multiplication

3x1 3x1 vector vector subtraction

Make more detailed the DSP, lookup table, and register needs of the whole accelerator, so we know if we need to use fixed point or alternative sincos

Determine how best to decrease the number of DSP blocks our system uses in the FPGA
