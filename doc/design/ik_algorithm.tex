\section{Inverse Kinematics Algorithm}
We first present an algorithm that takes as input $n$ homogenous transformation matrices $T^{i-1}_i$ for $i=1$ to $n$, and
the current position of our end effector in three-dimensional space, which is represneted as the vector $s$. 
The output of the algorithm is the Jacobian matrix for the current system configuration.\\
			\newpage

\begin{center}
\line(1,0){450}
\end{center}
\begin{algorithm}[H]
\DontPrintSemicolon
Let $J$ be a 1 x $n$ matrix, where each column is a vector in $\mathbb{R}^3$.
Let $z$ be the z-axis of the coordinate frame at the base of our robot appendage\;
\For{$i=1$ to $n$}{
	Let $v_i = zT^0_i$ be the axis of rotation or translation for joint $i$\;
	Let $p_i$ be the last column of $T^0_i$. This is the current position of joint $i$\;
	\eIf{joint $i$ is rotational}{
		Set column $i$ of $J$ to be $v_i \times (s - p_i)$\;
	}
	{
		Set column $i$ of $J$ to be $v_i$\;
	}
}
Return $J$
\caption{Jacobian($T^0_1, T^1_2, \ldots, T^{n-1}_n, s$)\label{IR}}
\begin{center}
\noindent\line(1,0){450}
\end{center}
\end{algorithm}
\vspace{2cm}

The following algorithm describes the Jacobian Damped Least-Squares method of solving the inverse kinematics problem. 
As input we are given a set of D-H parameters that fully define
the initial positions of the joints of our robot appendage, as well as a three-dimensional target position for our end effector. 
Formally, for each joint $i = 1 \ldots n$ we have D-H parameters $\theta_i, d_i, a_i, \alpha_i$, and
we call our target vector $t$. The final output of our algorithm is a set of updated D-H parameters that fully define
the required position of our joints such that the end effector position is sufficiently close to the target.
\newpage


\begin{center}
\line(1,0){450}
\end{center}
\begin{algorithm}[H]
Let $\epsilon$ be the desired accuracy of our final results\;
\For{$i = 1$ to $n$}{
	Calculate the homogenous transformation matrix $T^{i-1}_i$ for joint $i$ using the given D-H parameters\;
}
Let $T^0_n = \prod_{i=1}^n T^{i-1}_i$ be the full homogenous transformation matrix for the system\;
Let $s$ be the last column of $T^0_n$. This is the current position of our end effector\;
Let $e = t - s$ be the desired change in the position of our end effector\;
Let $J = JACOBIAN(T^0_1, T^1_2, \ldots, T^{n-1}_n, s)$ be the Jacobian matrix for the current system configuration\;
\While{the $l^2$ norm of $e$ is greater than $\epsilon$}{
	Let $J^T$ be the transpose of $J$\;
	Let $\lambda$ be a small positive constant\; 
	Let $I$ be the identity matrix\;
	Use row operations to determine the vector $f$ that satisfies the equation $(JJ^T+\lambda^2I)f = e$\;
	Let $\Delta \theta = J^T f$ be a vector whose $i$th component is a change in joint $i$'s angle. Note that
		if joint $i$ is translational along the unit vector $v_i$, then the joint ``angle" measures the distance
			moved in the direction $v_i$ and the $i$th component in $\Delta \theta$ will be a change in $d_i$\; 
	\For{$i = 1$ to $n$}{
		\eIf{joint $i$ is translational}{
			Set $d_i$ = $d_i + \Delta \theta [i]$\;
		}
		{
			Set $\theta_i = \theta_i + \Delta \theta[i]$\;
		}
		Recalculate $T^{i-1}_i$\;
	}
	Recalculate $T^0_n, s, e,$ and $J$ using our updated homogenous transformation matrices\;
}
Return the final set of D-H parameters currently specifying the positions of our joints\;
\begin{center}
\line(1,0){450}
\end{center}
\caption{JacobianTranspose($\{\theta_1, d_1, a_1, \alpha_1\}, \ldots, \{\theta_n, d_n, a_n, \alpha_n\}, t$)}
\end{algorithm}
