\documentclass[dvips, 12pt]{article}

% Any percent sign marks a comment to the end of the line

% Every latex document starts with a documentclass declaration like this
% The option dvips allows for graphics, 12pt is the font size, and article
%   is the style

\newcommand*{\TitleFont}{%
		\usefont{\encodingdefault}{\rmdefault}{b}{n}%
		\fontsize{16}{20}%
		\selectfont}

\newcommand{\HRule}{\rule{\linewidth}{0.5mm}}

\usepackage{epsfig}
\usepackage{graphicx}
\usepackage{url}
\usepackage[margin=1in]{geometry}

% These are additional packages for "pdflatex", graphics, and to include
% hyperlinks inside a document.


% These force using more of the margins that is the default style


% Everything after this becomes content
% Replace the text between curly brackets with your own

\title{{\Huge \bfseries IKSwift} \\ \Large \it Design of an Inverse Kinematics Accelerator \vspace{0.6cm}}

\author{\normalsize Yipeng Huang, Lianne Lairmore, Richard Townsend
	\\ \small \{yipeng, lairmore, rtownsend\}@cs.columbia.edu \vspace{0.6cm}}

\date{\today \vspace{2cm}}

% You can leave out "date" and it will be added automatically for today
% You can change the "\today" date to any text you like

\begin{document}
\maketitle

% This command causes the title to be created in the document
% \section{Overview}

In this project, we will build specialized hardware to tackle the inverse kinematics problem.

Inverse kinematics is widely used in robotics computing and in computer graphics. The problem takes as input a configuration of mechanical joints, which can be rotational or sliding, that are present in an arm or a leg. Then, it takes as input the limb's current shape and a target shape, solving for the required joint motions to get to the desired shape.

We will build a configurable solver on an FPGA, in hopes of speeding up solutions when compared to running the same algorithm on a regular CPU.
% \subsection*{Motivation}

Computers are increasingly embedded in the real world, often permanently attached to sensors and actuators. An example of such a system are robots. Computer systems that must coordinate with sensors and actuators are distinct from general purpose computers, and the desired hardware to support its applications will also be different.

Specialty hardware such as GPUs can support hard-hitting sensing algorithms, especially those that involve image processing. Equally important, but less well studied, is how computer hardware should adapt to support controlling actuators. Actuator algorithms have yet to appear in well known computer architecture research workloads. Early studies of computer architecture support for robotics show that general purpose CPUs suffer when running typical robotic workloads\cite{Caselli}.

Problems that arise when controlling actuators, such as kinematics, dynamics, obstacle avoidance, collision detection, have been found to occupy a large portion of computer runtime in robotics. Live measurements show 33-66 percent dedicated to embedded computer on robot. In particular, the inverse kinematic problem is interesting because it has features of two distinct workload categories: sparse matrix math and graph traversal\cite{dwarfs}. This project will focus on building hardware for inverse kinematics.


% \subsection*{Inverse Kinematics Algorithms}
The general inverse kinematics algorithm relies on the following concepts, which we present in order:

\subsubsection*{Homogeneous Transformations}
Forward and inverse kinematics operates on frames, which are as set of axes and coordinates that describe 3D space. Frames can be global or local. A local frame would be useful in describing the x, y, z positions of an object in space, along with the orientation (direction it is pointing) in space. We can transform from one frame to the next using homogeneous transforms, which are described as 4 by 4 matrices. For background information on homogeneous transforms refer to \cite{frames}.

\subsubsection*{Kinematic Chain}
Each joint of a robot has a coordinate frame; by convention without exception, revolute joints are represented as a rotation about the Z axis, and prismatic joints are translations along the Z axis. A collection of joints on an actuator form a kinematic chain.

\subsubsection*{Denavit-Hartenberg Parameters}
The standard notation for describing an actuator is to first describe the rotation about the Z axis by a joint angle, translate along by the Z axis by the link offset, translate along the X axis by the link length, and rotate by the X axis by the link twist.

\subsubsection*{Jacobian}
The matrix that relates the differential motion of joints to differential motion in cartesian space is called the Jacobian matrix. This matrix describes the velocity relationship between joints and the end of the actuator.\cite{jacobian}

\subsubsection*{Inverse Kinematics}
The inverse kinematics solution for any robot would be perfectly solved if the inverse of the Jacobian is available. This is because the inverse Jacobian describes the requisite joint motions to get any velocity in cartesian space.\cite{jacobian}

\subsubsection*{Jacobian Transpose and Jacobian Pseudoinverse Algorithms}
In reality, inverting a matrix is a costly operation, so the transpose or the pseudo-inverse of the Jacobian matrix are often used in inverse kinematics solvers.\cite{ik_intro} In this project, we will explore using dedicated hardware to compute the inverse Jacobian matrix. If that proves difficult, we will use one of the usual matrix inverse approximations.
% \section{Architecture}
For our project we will be using the FPGA as an accelerator for
inverse kinematics computations. Software running on the ARM processor 
on the SoCKit board will be driving the FPGA and will display its outputs on 
a monitor. The user should be allowed to specify a robot design via an XML file, which contains the Denavit-Hartenberg parameters for the robot. The software will supply target Cartesian coordinates 
to the accelerator, which will return updated joint configurations to 
move towards the given target coordinates. The software will use the 
joint configuration to update an image on the monitor. The resulting 
image should result in an animation of a appendage moving towards a target 
position.

% Figure
\begin{figure}[ht]
\center
\epsfig{figure=../figures/toolchain.eps,
width=0.6\columnwidth}
\caption{An architecture view of the software and hardware tools we will use for this design.}
\label{fig:toolchain}
\end{figure}
% %sine and cosine on FPGA
%Trigonometric Functions
%ALTFP_ATAN: Arctangent 34 cycle delay
%ALTFP_SINCOS: Trigonometric Sine/Cosine 36, 35 cycle delay

% \section{Project Plan}

\subsection{Milestones}

\begin{description}
\item[Completed Milestones] \hfill
\begin{enumerate}
\item Design block diagram of our system
\item Find C code that implements various inverse kinematics algorithms
\item Determine how to represent input and output with respect to the user (textual input, graphical output)
\end{enumerate}

\item[Milestone 1] \hfill
\begin{enumerate}
\item Write our own implementation of the damped least-squares algorithm in C
\item Design top-level module describing the interface between the hardware and software sections of our system
\item Design our joint peripheral device driver
\item Determine how best to decrease the number of DSP blocks our system uses in the FPGA
\end{enumerate}


\item[Milestone 2] \hfill 
\begin{enumerate}
\item Associate the different blocks in the diagram of our system with corresponding sections of C code in our implementation
\item Construct timing diagrams for each of our submodules
\item Begin coding the submodules of our system in SystemVerilog
\end{enumerate}

\item[Milestone 3] \hfill 
\begin{enumerate}
\item Full implementation of our system
\item Develop testbenches for the different modules in our system
\end{enumerate}

\item[Final Project Presentation] \hfill
\begin{enumerate}
\item Finish testing our system both in simulation and on the FPGA
\item Write up our final report and prepare our final presentation
\end{enumerate}

\end{description}

%Yipeng's notes
%
%
%milestone 1
%- software implementation
%- architecture
%- input output definition
%- hardware interface definition
%
%milestone 2
%- timing diagram
%- sub module design
%
%milestone 3
%- implementation
%
%final project presentation
%- test
%- evaluation
%- documentation


\subsection{D-H Parameter Homogeneous Transformation Block}

Frames are a set of axes and coordinates that describe 3D space. Frames can be global or local. A local frame would be useful in describing the x, y, z positions of an object in space, along with the orientation (direction it is pointing) in space. Each link in a robot appendage has a frame associated with it.

A moving joint that connects two links results in a change in reference frames between the two links preceding and following the joint. We can transform from one frame to the next using homogeneous transforms, which are described as 4 by 4 matrices. For background information on homogeneous transforms refer to \cite{frames}.

If we use the standard D-H parameters to describe the joint, this change in frames is a homogeneous transformation shown in Figure \ref{fig:dh_transform_equation}, which when multiplied out in full is the matrix shown in Figure \ref{fig:dh_transform_matrix}. 

% Figure
\begin{figure}[ht]
\center
\epsfig{figure=../figures/dh_transform_equation.eps,
width=0.4\columnwidth}
\caption{The transformation between two frames linked by a joint, using D-H parameter variables.}
\label{fig:dh_transform_equation}
\end{figure}

Recall that revolute joints are represented as a rotation of joint angle $\alpha$ about the $Z$ axis, and prismatic joints are translations by link offset $a$ along the $Z$ axis. To align the two coordinates frames, we translate along the X axis by the link length $d$, and rotate by the X axis by the link twist $\theta$.

% Figure
\begin{figure}[ht]
\center
\epsfig{figure=../figures/dh_transform_matrix.eps,
width=0.4\columnwidth}
\caption{The full matrix describing the transformation between two frames linked by a joint.}
\label{fig:dh_transform_matrix}
\end{figure}

We can calculate this homogenous transformation using a dedicated hardware block in the FPGA. Figure \ref{fig:dh_transform_block} is the dataflow diagram for a hardware block that calculates all the elements in the transformation matrix.

% Figure
\begin{figure}[ht]
\center
\epsfig{figure=../figures/dh_transform_block.eps,
width=0.4\columnwidth}
\caption{The dataflow diagram for a hardware block that calculates the elements in a homogenous transform matrix.}
\label{fig:dh_transform_block}
\end{figure}

% The D-H transformation block would require four floating point sine or cosine pipelines and six multipliers.

% For the sine and cosine functions we will use the ALTFP\_SINCOS IP block from Altera. These blocks generate a result within 35 or 36 clock cycles, and require 28 DSP blocks, 10,986 lookup tables, and 18,834 registers.

% We will also consider alternative ways to compute sine and cosine.


% citations begin here
\bibliographystyle{ieeetr}
\bibliography{design}
% \bibliography{ref/refs}


\end{document}
