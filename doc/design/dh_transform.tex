\section{D-H Parameter Homogeneous Transformation Block}

A moving joint results in a change in reference frames between the two links preceding and following the joint. If we use the standard D-H parameters to describe the joint, this change in frames is a homogeneous transformation shown in Figure \ref{fig:dh_transform_equation}, which when multiplied out in full is the matrix shown in Figure \ref{fig:dh_transformation_matrix}.

% Figure
\begin{figure}[ht]
\center
\epsfig{figure=../figures/dh_transform_equation.eps,
width=0.6\columnwidth}
\caption{The transformation between two frames linked by a joint, using D-H parameter variables.}
\label{fig:dh_transform_equation}
\end{figure}

% Figure
\begin{figure}[ht]
\center
\epsfig{figure=../figures/dh_transform_matrix.eps,
width=0.4\columnwidth}
\caption{The full matrix describing the transformation between two frames linked by a joint.}
\label{fig:dh_transform_matrix}
\end{figure}

Recall that revolute joints are represented as a rotation of joint angle $\alpha$ about the $Z$ axis, and prismatic joints are translations by link offset $a$ along the $Z$ axis. To align the two coordinates frames, we translate along the X axis by the link length $d$, and rotate by the X axis by the link twist $\theta$.

We can calculate this homogenous transformation using a dedicated pipeline in the FPGA. Figure \ref{fig:dh_transformation_block} is the dataflow diagram for a hardware block that calculates all the elements in the transformation matrix.

% Figure
\begin{figure}[ht]
\center
\epsfig{figure=../figures/dh_transform_block.eps,
width=0.4\columnwidth}
\caption{The dataflow diagram for a hardware block that calculates the elements in a homogenous transform matrix.}
\label{fig:dh_transform_block}
\end{figure}

The D-H transformation block would require four sine or cosine pipelines and six multipliers. These blocks require 28 DSP blocks, 10986 lookup tables, 18834 registers.

% Figure
\begin{figure}[ht]
\center
\epsfig{figure=../figures/dh_transform_pipeline.eps,
width=0.6\columnwidth}
\caption{The homogenous transform matrix blocks are chained together to calculate the full transformation matrix of the joint chain.}
\label{fig:dh_transform_block}
\end{figure}
